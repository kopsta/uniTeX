\usepackage{hyperref}
\usepackage[utf8]{inputenc} % universelle Zeichenlesung
\usepackage[ngerman]{babel} % Deutsche Bezeichnungen für tableofcontents, figure, etc.
\usepackage[T1]{fontenc} % um nochmal sicher zu gehen, dass alle Zeichen korrekt übernommen und benutzt werden 
\usepackage{lmodern}
\usepackage{graphicx} % Einbetten von Bildern und eps
\usepackage{amsmath} % mehr Mathe-Pakete braucht ein Chemiker nicht
\usepackage{pdfpages} % ermöglicht das Einbetten von ganzen PDFs und einzelnen Seiten daraus auf ganze Seiten
\usepackage{subscript} % 
\usepackage{siunitx} %
\usepackage{textcomp} % für \textcelsius
\usepackage{subfigure} % für \ Subfigures yeah!
\usepackage{csvsimple} % für Tabellen in csv
\usepackage{acronym}

%CHEMIE

\usepackage{chemstyle} % for chemstyle/chemscheme
\usepackage[version=3]{mhchem} % gut für Strukturformeln und Reaktionsgleichungen mit vielen Abkürzungen für Standardanwendungen
\usepackage{chemscheme} % Extra Caption für Schemata
\usepackage{chemfig} % Komplexe Lewis-Strukturen und ganze Reaktiondiagramme


%LAYOUT

\usepackage[onehalfspacing]{setspace} % 1,5er Zeilenabstand
\usepackage{parskip} % Verhindert den Einzug nach Einbetten von inline-Elementen und setzt 
\usepackage{float}	% Behebt das float Problem beim Einbetten von inline-Elementen (z.B. fig, equation, table) und ermöglicht genau Positionierung mit "[H]" = besser als "h"
\usepackage{microtype} % bessere Lesbarkeit (allerdings nicht unbedingt bemerkbar ohne Vergleich)
\usepackage{tabularx} %

\usepackage[hcentering,bindingoffset=10mm]{geometry} % 
	\geometry{a4paper, top=25mm, bottom=25mm} % bitte die Ränder und DIN-Format anpassen, falls gewünscht
\usepackage[super, comma, numbers, square, sort]{natbib}
	\floatstyle{plaintop}
	\restylefloat{table}

\usepackage{fancyhdr}
	\renewcommand{\headrulewidth}{0.1pt} %obere Trennlinie
	\renewcommand{\footrulewidth}{0pt} %untere Trennlinie
	\fancyfoot[C]{ {\thepage} }
\usepackage{geometry}
	\geometry{a4paper, top=20mm, left=35mm, right=35mm, bottom=30mm}